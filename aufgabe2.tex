\documentclass{article}
\usepackage[ngerman]{babel}
\date{\today}
\author{Julian Egli}
\title{CS102 \LaTeX  \ \"Ubung}
\begin{document}
\maketitle

\section{Ich war hier...}
No es sch\"ons Wochenend. Gruess Thomas R\"olli (Abrahma)

\section{Das ist der erste Abschnitt}
das ist ein Test

\section{Tabelle}
Beispiel f\"ur eine Tabelle:
\begin{table}[h]
\centering
\begin{tabular}{c|c|c|c}
 & yxyz & yxzyx & xzyxz \\
\hline L1  & 233 & 274 & 128 \\
L2 & 639 & 242 & 758 \\
L3 & 534 & 678 & 133 \\
\end{tabular}
\caption{Beschreibung}
\end{table}

\section{Formeln}
\subsection{Pythagoras}
Der Satz des Pythagoras errechnet sich wie folgt: $a^2 + b^2 = c2$. Daraus k\"onnen
wir die L\"ange der Hypothenuse wie folgt berechnen: $c = \sqrt{a^2 + b^2}$
\subsection{Summen}
Wir k\"onnen auch die Formel f\"ur eine Summe angeben:
\newline
\center $s = \sum\limits_{i=1}^{n}i = \frac{n * (n + 1)}{2}$\begin{flushright} (1) \end{flushright}


\end{document}